\documentclass{article} 
\usepackage{graphicx}
\usepackage{here}
\usepackage{hyperref}
\usepackage{polski}
\usepackage{color}
\usepackage[utf8]{inputenc}

\usepackage{titlesec}
\titlespacing*{\section}
{0pt}{5.5ex plus 1ex minus .2ex}{4.3ex plus .2ex}

\graphicspath{ {C:/image}}
\title{.NET Micro Framework
STM32F4 Discovery}
\date{2016.4.21}
\author{Wojciech Duda}
\begin{document} 
\maketitle 
\pagenumbering{gobble} 
\newpage
\pagenumbering{arabic}
\tableofcontents
\newpage
\section{Teoria}
Rdzeń CortexM4F wykorzystuje architekturę ARMv7M. Pod względem organizacji pamięci jest to architektura harwardzka, tzn. pamięć zawierająca kod programu (Flash) i pamięć danych (SRAM) są rozdzielone i dostęp do nich odbywa się poprzez osobne magistrale.

\begin{figure}[H]
\includegraphics[width=\textwidth]{image/1.jpg}
\caption{Opis urządzenia}
\end{figure}
\section{Instalacja}
\subsection{Narzędzia:}
\begin{itemize}
\item mikrokontroler STM32F4 Discovery
\item kable USB Micro oraz USB Mini
\item Visual studio
\item \href{http://www.st.com/content/st_com/en/products/embedded-software/development-tool-software/stsw-link004.html}{STM32 ST-LINK Utility (kliknij aby pobrać)}
\item \href{www.codeplex.com/Download?ProjectName=netmf4stm32&DownloadId=471395}{sterwonik USB (kliknij aby pobrać)}
\item \href{www.codeplex.com/Download?ProjectName=netmf4stm32&DownloadId=471396}{bootloader oraz pliki hex (kliknij aby pobrać)}
\item \href {netmf.codeplex.com/releases/view/91594}{.NET MicroFramework SDK (kliknij aby pobrać)}
\end{itemize}
\subsection{Konfiguracja}
\begin{enumerate}
\item Pobierz pliki z punktu 2.1.
\item Zainstaluj STM32 ST-LINK Utility, oraz SDK, resztę plików rozpakuj.
\item Podłącz kabel USB Mini (do wejścia oznaczonego jako “Złącze USB” na Rysunku 1.)
\item Włącz STLINK Utility , a następnie połącz się z stm32f4 poprzez przycisk: “Connect to the = target”

\begin{figure}[H]
\includegraphics[width=\textwidth]{image/2.jpg}
\caption{STLINK Utility}
\end{figure}

\item Następnie wybierz Target \textgreater Erase Chip oraz Target\textgreater Erase Sectors, wybierz wszystkie i potwierdź. Wybierz Target \textgreater Program…, wybierz ścieżkę Tinybooter.hex a następnie wybierz start. Zresetuj mikrokontroler poprzez przycisk zerujący.

\begin{figure}[H]
\includegraphics[width=\textwidth]{image/3.jpg}
\caption{Programowanie debuggera}
\end{figure}

\item Jeżeli wszystko przebiegło prawidłowo powinny zapalić się 3 diody użytkowe. Podłącz kabel micro USB (jak na rysunku 4).

\begin{figure}[H]
\includegraphics[width=\textwidth]{image/4.jpg}
\caption{Podłączony STM32F4 kablami mikro i mini USB}
\end{figure}

\item Przejdź do “urządzenia i drukarki”. Tam w obszarze“ nieokreślone” kliknij prawym przyciskiem myszy w “STM .Net Test” i wybierz właściowości.

\begin{figure}[H]
\includegraphics[width=\textwidth]{image/5.jpg}
\caption{Urządzenia i drukarki}
\end{figure}

\item Wejdź w sprzęt \textgreater właściowości \textgreater zmień ustawienia \textgreater sterownik \textgreater Aktualizuj sterownik…

\begin{figure}[H]
\includegraphics[width=\textwidth]{image/6.jpg}
\caption{Instalacja sterownika krok 1}
\end{figure}
\begin{figure}[H]
\includegraphics[width=\textwidth]{image/7.jpg}
\caption{Instalacja sterownika krok 2}
\end{figure}

\item Wybierz “Przeglądaj mój komputer w poszukiwaniu oprogramowania sterownika” i wybierz ścieżkę gdzie rozpakowałeś na początku sterownik. Podczas instalacji ignoruj ostrzeżenia.
\item Uruchom MFDeploy. Wybierz Device: USB. Naduś przycisk Ping. Następnie drugie od góry Browse... , wybierz ścieżkę pozostałych dwóch plików hex: ER\_CONFIG.hex, ER\_FLASH.hex oraz wybierz Deploy.

\begin{figure}[H]
\includegraphics[width=\textwidth]{image/8.jpg}
\caption{MF Deploy}
\end{figure}
\item Włącz Visual studio utwórz nowy projekt i wybierz C\# \textgreater Micro Framework \textgreater Console Application.

\begin{figure}[H]
\includegraphics[width=\textwidth]{image/9.jpg}
\caption{Tworzenie projektu}
\end{figure}
\item W utworzonym projekie, w Solution Explorer kliknij prawym przyciskiem myszy na projekt i wybierz “Properties”. Tam wybierz .NET Micro Framework i Transport ustaw na USB.

\begin{figure}[H]
\includegraphics[width=\textwidth]{image/10.jpg}
\caption{Konfigurowanie Visual Studio}
\end{figure}
\end{enumerate}

\section{Przycisk}
\subsection{Klasa InterruptPort}
Klasa zdefiniowana w przestrzeni nazw Microsoft.SPOT.Hardware.
\subsubsection{Referencje}
\begin{itemize}
\item Microsoft.SPOT.Hardware
\end{itemize}
\subsubsection{Konstruktor}
 InterruptPort (Pin \textcolor{red}{portId}, bool \textcolor{red}{glitchFilter}, ResistorMode \textcolor{red}{resistor},\newline InterruptMode \textcolor{red}{interrupt})
\begin{itemize}
\item \textcolor{red}{portId} - identyfikator portu.
\item \textcolor{red}{glitchFilter}, - obsługa filtra błędów: true -włączony, false-wyłączony
\item \textcolor{red}{resistor} - tryb rezystora, który określa stan domyślny dla portu.
\item \textcolor{red}{interrupt} - tryb przerwania, który określa warunki wymagane do\newline generowania przerwania.
\end{itemize}
\subsubsection{Funkcje}
bool Read () - zwraca aktualną wartość portu.

\subsection{Program}

Aby napisać program z użyciem przycisku trzeba najpierw utworzyć dla niego obiekt:\newline \newline
InterruptPort button = 
new InterruptPort(\space \textcolor{red}{ (Cpu.Pin)0}, \space \textcolor{red}{ false},\space \textcolor{red}{ Port.ResistorMode.PullDown}, \space \textcolor{red}{ Port.InterruptMode.InterruptEdgeLevelHigh});
\begin{itemize}
\item \textcolor{red}{(Cpu.Pin)0} - Przycisk znajduję się na zerowym pinie portu A, każdy port ma 16 pinów. Port A jest pierwszym portem, więc 16*0+0=0.
\item \textcolor{red}{false} - wyłączona obsługa filtru błędów
\item \textcolor{red}{Port.ResistorMode.PullDown} - rezystor ustawiony na pulldown (Kiedy przycisk nie jest aktywny, zwracana jest wartość logiczna 0)
\item \textcolor{red}{Port.InterruptMode.InterruptEdgeLevelHigh} - włącza przerwanie kiedy wartość portu jest wysoka.
\end{itemize}
Po utworzeniu obiektu dla przycisku można też zdeklarować inne obiekty potrzebne do programu(np. led opisane w punkcie 4). Po tych czynnościach można odczytać wartość portu przycisku(czy przycisk jest wduszony) za pomocą:\newline\\
\textcolor{red}{while(true)}\\
 if (\textcolor{red}{button.Read()} == true)\newline
 \{kod programu(np. włączanie diód) \}\\
 else\\
 \{kod programu(np. wyłączenie diód) \}\\
 
 \begin{itemize}
\item \textcolor{red}{while(true)} - Pętla sterująca, dzięki niej jest ciągle sprawdzana wartość portu przycisku.
\item \textcolor{red}{button.Read()} - Zwraca aktualną wartość portu przycisku.
\end{itemize}
 
 
\section{LED}
\subsection{Klasa OutputPort}
Klasa zdefiniowana w przestrzeni nazw Microsoft.SPOT.Hardware.
\subsubsection{Referencje}
\begin{itemize}
\item Microsoft.SPOT.Hardware
\end{itemize}
\subsubsection{Konstruktor}
 OutputPort (Pin \textcolor{red}{portId},bool \textcolor{red}{initialState})
\begin{itemize}
\item \textcolor{red}{portId} - identyfikator portu.
\item \textcolor{red}{initialState} - stan początkowy na porcie po aktywacji.
\end{itemize}
\subsubsection{Funkcje}
void Write(bool \textcolor{red}{state}) - wpisuje wartość do portu.
\begin{itemize}
\item \textcolor{red}{state} - wartość wpisywana do portu.
\end{itemize}
\subsection{Program}
Aby napisać program z użyciem diod trzeba dla każdej używanej diody stworzyć obiekt:\newline \newline
OutputPort led = new OutputPort(\space \textcolor{red}{(Cpu.Pin)x},\space \textcolor{red}{ false})
\begin{itemize}
\item \textcolor{red}{(Cpu.Pin)x} - x- przyjmuje wartości od 60-63 diody znajdują się na \newline końcowych pinach portu D, każdy port ma 16 pinów, więc 16*3+12=60 oraz 16*3+15=63.(niebieska-63, czerwona-62, pomarańczowa-61, zielona-60)
\item \textcolor{red}{false} - stan początkowy diod - wyłączony
\end{itemize}
Po utworzeniu obiektów dla diod, można zająć się ich obsługą:
\\ \textcolor{red}{while(true)}\\
            \{\\
                \textcolor{red}{led.Write(true);}\\
                \textcolor{green}{for (int i = 0; i < 100000; i++) { }}\\
                \textcolor{red}{led.Write(false);}\\
                \textcolor{green}{for (int i = 0; i < 100000; i++) { }}\\
\}\\
\begin{itemize}
\item \textcolor{red}{while(true)} - Pętla sterująca, dzięki niej diody będą ciągle zmieniały swój stan.
\item \textcolor{red}{led.Write(true);} - Włączenie diody.
\item \textcolor{red}{led.Write(false);} - Wyłączenie diody.
\item \textcolor{green}{for (int i = 0; i < 100000; i++) { }} - pętle opóźniające, dzięki nim diody wolniej zmieniają stan.
\end{itemize}

\section{PWM}
\subsection{Klasa PWM}
Klasa zdefiniowana w przestrzeni nazw Microsoft.SPOT.Hardware. 
\subsubsection{Referencje}
\begin{itemize}
\item Microsoft.SPOT.Hardware.PWM
\item Microsoft.SPOT.Hardware
\end{itemize}
\subsubsection{Konstruktor}
PWM (PWMChannel \textcolor{red}{channel}, Double \textcolor{red}{frequency\_Hz}, Double \textcolor{red}{dutyCycle},  \textcolor{red}{bool invert})
\begin{itemize}
\item \textcolor{red}{channel} - kanał PWM
\item \textcolor{red}{frequency\_Hz} - Częstotliwość impulsów w Hz.
\item \textcolor{red}{dutyCycle} - Określa ile całkowitego czasu jest przeznaczonego na prace jako wartość od 0.0 do 1.0(0-100\%).
\item \textcolor{red}{invert} - Wartość, która wskazuje, czy wyjście jest odwrócone.
\end{itemize}
\subsubsection{Atrybuty}
double DutyCycle - Pobiera lub ustawia cykl pracy impulsu jako wartość od 0.0 do 1.0.
\subsubsection{Funkcje}
void Start () - Uruchamia port PWM na nieokreślony czas.
\subsection{Program}
Aby napisać program obsługujący diody za pomocą PWM trzeba dla każdej używanej diody stworzyć obiekt:\newline \newline
 var led = new PWM(\space \textcolor{red}{ Cpu.PWMChannel.PWM\_x },\space \textcolor{red}{ 300},\space \textcolor{red}{ 0}, \space \textcolor{red}{ false});
\begin{itemize}
\item \textcolor{red}{Cpu.PWMChannel.PWM\_x} - x- przyjmuje wartości od 0 do 3. Oznaczając kanały PWM od 0 do 3(0-zielona, 1-pomarańczowa, 2-Czerwona, 3-Niebieska).
\item \textcolor{red}{300} - Za niska częstotliwość może spowodować,] że jasność diod nie zdążyć się zmienić.
\item \textcolor{red}{0} - przyjmuje 0\% czasu cyklu pracy.
\item \textcolor{red}{false} - wyjście ustawione jako nieodwrócone.
\end{itemize}
Oraz dla każdej trzeba wywołać funkcję Start:\\\\
led.Start();
Następnie trzeba zdefiniować kod programu(przykładowe zastosowanie):\\\\
\textcolor{red}{while (true)}\\
for (int i = 0; i <= 10; i++)\\
\{\\
\textcolor{red}{led.DutyCycle = 1 - ((double)i / 10);}\\ 
\textcolor{red}{Thread.Sleep(1000);}\\ 
\}
\begin{itemize}
\item \textcolor{red}{while (true)} - Pętla sterująca, dzięki niej diody będą ciągle zmieniały swój stan.
\item \textcolor{red}{led.DutyCycle = 1 - ((double)i / 10);} - zmiana mocy świecenie diody.
\item \textcolor{red}{Thread.Sleep(1000);} - uśpienie wątku na sekundę(1000 milisekund). Dzięki czemu diody co sekundę zmeiniają moc świecenia
\end{itemize}

\section{Zegar czasu rzeczywistego}
\subsection{Klasa DateTime} 
Klasa zdefiniowana w przestrzeni nazw Microsoft.SPOT.
\subsubsection{Atrybuty}
\begin{itemize}
\item DateTime.Now.Second - zwraca sekundy z aktualnego czasu. Przyjmuje wartości od 0 do 59.
\item DateTime.Now.Ticks - zwraca aktualną ilość przeskoków zegara. 
\end{itemize}
\subsection{Program}
Przy samej obsłudze czasu nie trzeba tworzyć obiektów. Przykładowy kod w funkcji main:\\\\
 \textcolor{red}{while (true)}\\
            \{\\
                if (\textcolor{red}{DateTime.Now.Second\% 2==0})\\
				\{Kod programu\}\\
	\}
\begin{itemize}
\item \textcolor{red}{while (true)} -Pętla sterująca, dzięki niej program się nie zakończy.
\item \textcolor{red}{DateTime.Now.Second\% 2==0} - Jeżeli sekundy z aktualnego czasu nie są nieparzyste, wykonaj Kod programu.
\end{itemize}

\section{SPI-Akcelerometr}
\subsection{Klasa SPI}
Klasa zdefiniowana w przestrzeni nazw Microsoft.SPOT.Hardware. 
\subsubsection{Referencje}
\begin{itemize}
\item Microsoft.SPOT.Hardware
\end{itemize}
\subsubsection{Konstruktor}
SPI (\textcolor{red}{Config})
\begin{itemize}
\item \textcolor{red}{Config} - Konfiguracja interfejsu SPI
\end{itemize}
\subsubsection{Funkcje}
-void Write (byte[] \textcolor{red}{writeBuffer}) - wpisuje blok danych do interfejsu.
\begin{itemize}
\item \textcolor{red}{writeBuffer} - buffor, który zostanie zapisany do interfejsu.
\end{itemize}
-void WriteRead ( byte[] \textcolor{red}{writeBuffer},ref byte[] \textcolor{red}{readBuffer})
\begin{itemize}
\item \textcolor{red}{writeBuffer} - buffor, który zostanie zapisany do interfejsu.
\item \textcolor{red}{readBuffer} - buffor do którego zostaną zapisane dane odczytane z \newline interfejsu.
\end{itemize}

\newpage
\subsection{Klasa SPI.Configuration}
Klasa zdefiniowana w przestrzeni nazw Microsoft.SPOT.Hardware. 
\subsubsection{Referencje}
\begin{itemize}
\item Microsoft.SPOT.Hardware
\end{itemize}
\subsubsection{Konstruktor}
SPI.Configuration (Pin \textcolor{red}{ChipSelect\_Port}, bool \textcolor{red}{ChipSelect\_ActiveState}, \newline UInt16 \textcolor{red}{ChipSelect\_SetupTime}, UInt16 \textcolor{red}{ChipSelect\_HoldTime},  bool \textcolor{red}{Clock\_IdleState}, bool \textcolor{red}{Clock\_Edge}, UInt16 \textcolor{red}{Clock\_Rate}, SPI\_module \textcolor{red}{SPI\_mod})
\begin{itemize}
\item \textcolor{red}{ChipSelect\_Port} - Port wybranego czipu.
\item \textcolor{red}{ChipSelect\_ActiveState} - Stan aktywny dla portu wybranego czipu. Jeżeli prawda- port będzie ustawiany na wysoki w momencie dostępu do czipu, jezeli fałsz- port będzie ustawiany na niski w momencie dostępu do czipu.
\item \textcolor{red}{ChipSelect\_SetupTime} - Czas pomiędzy wybraniem urządzenia a momentem kiedy zegar rozpocznie transakcje. 
\item \textcolor{red}{ChipSelect\_HoldTime} - Określa, jak długo port czipu musi zostać w stanie aktywnym po zakończeniu transakcji czytania lub pisania.
\item \textcolor{red}{Clock\_IdleState} - Stan bezczynności zegara. Jeśli prawda- sygnał zegara SPI zostanie ustawiony na wysoki, gdy urządzenie jest w stanie spoczynku. Jeśli fałsz- sygnał zegara SPI zostanie ustawiony na niski, gdy urządzenie jest w stanie bezczynności. 
\item \textcolor{red}{Clock\_Edge} - Jeśli prawda- dane są próbkowane na zboczu wznoszącym zegara SPI. Jeśli fałsz- dane są próbkowane na zboczu opadającym zegara SPI.
\item \textcolor{red}{Clock\_Rate} - Częstotliwość zegara SPI w kHz.
\item \textcolor{red}{SPI\_mod} - Magistrala SPI używana do transakcji.
\end{itemize}
\newpage
\subsection{Program}
W klasie programu trzeba stworzyć obiekt:\\\\
static SPI MySPI = null;\\\\
Trzeba zdefiniować jedną metodę:\\\\
public static void WriteRegister( \textcolor{red}{byte register},  \textcolor{red}{byte data})\\
\{\\
\textcolor{red}{byte[] tx\_data = new byte[2];}\\
tx\_data[0] = (byte)(register | 0x00);\\
tx\_data[1] = data;                \\   
\textcolor{red}{MySPI.Write(tx\_data);}\\
\}\\
\begin{itemize}
\item \textcolor{red}{byte register} - Adres rejestru.
\item \textcolor{red}{byte data} - Wartość rejestru.
\item \textcolor{red}{byte[] tx\_data = new byte[2];} - Tablica, która będzie wpisana do SPI. W 0 elemencie przechowuje adres rejestru w 1 elemencie przechowuje wartość rejestru.
\item \textcolor{red}{MySPI.Write(tx\_data);} - wpisuje blok danych do akcelerometru.
\end{itemize}
Oraz jedną funkcję, zwracającą wartość z rejestru:\\\\
public static byte ReadRegister(\textcolor{red}{byte register})\\
 \{\\
\textcolor{red}{byte[] tx\_data = new byte[2];}\\
\textcolor{red}{byte[] rx\_data = new byte[2]};\\
 tx\_data[0] = (byte)(register | 0x80);\\
 tx\_data[1] = 0;\\
 \textcolor{red}{MySPI.WriteRead(tx\_data, rx\_data);}\\
 return rx\_data[1];   \\                   
\}
\begin{itemize}
\item \textcolor{red}{byte register} - Adres rejestru.
\item \textcolor{red}{byte[] tx\_data = new byte[2];} - Tablica, która będzie wpisana do SPI. W 0 elemencie przechowuje adres rejestru w 1 elemencie przechowuje wartość rejestru.
\item \textcolor{red}{byte[] rx\_data = new byte[2]}; - Tablica, która będzie przechowywać wartości odczytane z rejestru.
\item \textcolor{red}{MySPI.WriteRead(tx\_data, rx\_data);} - wpisuje i odczytuje bloki danych z akcelerometru.
\end{itemize}
W funkcji Main trzeba stworzyć obiekt SPI.Configuration:\\\\
SPI.Configuration MyConfig =
new SPI.Configuration(\space \textcolor{red}{ (Cpu.Pin)67},\space \textcolor{red}{ false},\space \textcolor{red}{ 0}, \space \textcolor{red}{ 0},\space \textcolor{red}{ true},\space \textcolor{red}{ true},\space \textcolor{red}{ 1000},\space \textcolor{red}{ SPI.SPI\_module.SPI1})
\begin{itemize}
\item \textcolor{red}{(Cpu.Pin)67} - SPI znajduje się na trzecim pinie portu E, czyli 16*4+3 = 67.
\item \textcolor{red}{false} - Port będzie ustawiany na niski w momencie dostępu do czipu.
\item \textcolor{red}{0} - Natychmiastowe rozpoczęcie transakcji w momencie wybrania urządzenia
\item \textcolor{red}{0} - Brak stanu aktywności po zakończeniu transakcji czytania lub pisania.
\item \textcolor{red}{true} - Sygnał zegara SPI zostanie ustawiony na wysoki, gdy urządzenie jest w stanie spoczynku.
\item \textcolor{red}{true} -  Dane są próbkowane na zboczu wznoszącym zegara SPI.
\item \textcolor{red}{1000} -  Częstotliwość zegara SPI jest równa 1000 kHz
\item \textcolor{red}{SPI.SPI\_module.SPI1} - Magistrala SPI 1. 
\end{itemize}
Trzeba zdefiniować globalny obiekt MySPI:\\\\
MySPI = new SPI(MyConfig);\\\\
Uaktywnić akcelerometr:\\\\
WriteRegister(\space \textcolor{red}{0x20},\space \textcolor{red}{0xC7})
\begin{itemize}
\item \textcolor{red}{0x20} -  W kodzie binarnym jest równe 0010 0000, pierwsze 00 oznacza tryb pracy zapisu, a reszta jest adresem rejestru.
\item \textcolor{red}{0xC7} -  W kodzie binarnym jest równe 11000111: 
\newline 1 - szybkość danych wyjściowych 400Hz(zero oznacza -100Hz)
\newline 1 - ustawienie urządzenia w trybie aktywnym
\newline 0 - wartości muszą być zero aby określone były zakresy X,Y,Z.
\newline 00 - normlany tryb.
\newline 111 - oznacza włączenie kolejno Z,Y,X.
\end{itemize}
Aby odczytytać wartości akcelerometru, trzeba posłużyć się funckją \\ReadRegister(\textcolor{red}{X}), gdzie X może przyjąć jedną z wartości:
\begin{itemize}
\item \textcolor{red}{0x2D } - Rejestr z wartością Z.
\item \textcolor{red}{ 0x29 } - Rejestr z wartością X.
\item \textcolor{red}{ 0x2B } - Rejestr z wartością Y.
\end{itemize}

\section{Timer}
\subsection{Klasa Timer}
Klasa zdefiniowana w przestrzeni nazw System.Threading.
\subsubsection{Konstruktor}
Timer(TimerCallback \textcolor{red}{callback}, object \textcolor{red}{state}, uint \textcolor{red}{dueTime}, uint \textcolor{red}{period})
\begin{itemize}
\item \textcolor{red}{callback} - nazwa metody, która ma być wykonywana.
\item \textcolor{red}{state} - obiekt z informacjami wykorzysytwanych w metodzie callback lub null.
\item \textcolor{red}{dueTime} - opóźnienie, z jakim będzie wywoływać się metoda callback, podane w millisekundach.
\item \textcolor{red}{period} - czas między wywołananimy metody callback, podany w millisekundach.
\end{itemize}
\subsection{Program}
Aby uzyskać timer trzeba zdeklarować specjalną metodę, która będzie wywoływana:\\\\
 public static void  \textcolor{red}{nazwa}(\textcolor{red}{object state})
\{
Kod programu
\}
\begin{itemize}
\item \textcolor{red}{nazwa} - zdefiniowana przez programiste nazwa wywowywanej metody przez Timer.
\item \textcolor{red}{object state} - dodatkowy obiekt z infromacjami wykorzystywanymi w metodzie.
\end{itemize}
Pozostało utworzyć obiekt timer:\\\\
Timer timer = new System.Threading.Timer(\space \textcolor{red}{ funTimer},\space \textcolor{red}{ null},\space \textcolor{red}{ 0},\space \textcolor{red}{ 1000});
\begin{itemize}
\item \textcolor{red}{funTimer} - Metoda, która ma być wykonywana.
\item \textcolor{red}{null} - Brak obiektu z infromacjami wykorzystywanymi w metodzie FunTimer.
\item \textcolor{red}{0} - Brak opóźnienia wywołania metody FunTimer.
\item \textcolor{red}{1000} - Czas, co ile będzie wywoływać się metoda Funtimer(1 sekunda)
\end{itemize}
\end{document}

